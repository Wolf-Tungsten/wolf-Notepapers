\documentclass[UTF8]{ctexart}
\usepackage{fontspec}
\defaultfontfeatures{Mapping=tex-text}
\usepackage{xunicode}
\usepackage{xltxtra}
%\setmainfont{???}
\usepackage{amsmath}
\usepackage{amsfonts}
\usepackage{amssymb}
\usepackage{graphicx}
\usepackage[left=2cm,right=2cm,top=2cm,bottom=2cm]{geometry}
\author{高睿昊}
\title{\zihao{0}\heiti Git学习笔记}
\newenvironment{myquote}
  {\begin{quote}\kaishu\zihao{-5}}
  {\end{quote}}
  
  \usepackage{listings}
  \lstset{
flexiblecolumns}
\begin{document}
\maketitle

\section{安装git}
一提到git,大佬们常说:
\begin{myquote}
\kaishu “Git是目前世界上最先进的分布式版本控制系统”
\end{myquote}
所以我们必须理解Git实际上是一个软件,同样需要安装。

在Ubuntu上安装Git的命令:\verb!$sudo apt-get install git! 。(Windows和Mac上面的安装很简单,问度娘就好了)
安装完成以后还需要配置账户信息
\begin{lstlisting}[language=bash]
$ git config --global user.name "YourName"
$ git config --global user.email "email@example.com"
\end{lstlisting}

\section{创建版本库}
在需要Git管理的目录下执行命令:
\begin{lstlisting}[language=bash]
$ git init
\end{lstlisting}
对于需要Git管理的目录需要使用add命令进行处理
\begin{lstlisting}[language=bash]
$ git add 文件名或者目录名
\end{lstlisting}
最后使用commit命令即可提交到版本库
\begin{lstlisting}[language=bash]
$ git commit -m “用于识别的附加在提交后面的消息”
\end{lstlisting}

\section{Git时光机}
\textbf{git status}命令可以让我们时刻掌握仓库当前的状态,上面的命令告诉我们,readme.txt被修改过了,但还没有准备提交的修改。

\textbf{git diff}顾名思义就是查看difference,显示的格式正是Unix通用的diff格式,可以从上面的命令输出看到,我们在第一行添加了一个“distributed”单词。

当我们对文件修改后再提交时,也要经过添加新的文件的add和commit命令。

\subsection{版本回退}
在Git中,可以用git log命令查看提交的版本信息,例如在编写个笔记过程中使用的log是这样的:
\begin{verbatim}
commit a0a5fe4e08f77bc5bf609b27575206cb7cb1ccb2
Author: Acinonyx Tungsten <gaoruihao@outlook.com>
Date:   Thu Feb 9 19:54:30 2017 +0800

    1

commit 047711edf118019cd76d9628d3d61b6022f6b008
Author: Acinonyx Tungsten <gaoruihao@outlook.com>
Date:   Thu Feb 9 08:39:34 2017 +0800

    add a line

commit 41320d5047bf318ab067352bfaa5ce658646d188
Author: Acinonyx Tungsten <gaoruihao@outlook.com>
Date:   Thu Feb 9 08:33:59 2017 +0800

    first time
\end{verbatim}

这样看起来比较烦的话可以用\verb! $ git log --pretty=oneline!,得到的结果是这样的:
\begin{verbatim}
wolf-tungsten@wolftungsten-ThinkPad-T460p:~/NotePapers$ git log --pretty=oneline
a0a5fe4e08f77bc5bf609b27575206cb7cb1ccb2 1
047711edf118019cd76d9628d3d61b6022f6b008 add a line
41320d5047bf318ab067352bfaa5ce658646d188 first time
\end{verbatim}

前面一串串的数字就是十六进制表示的版本号了(官方名称是commit id)。当我们想回退到上一个版本的时候,就可以用命令\verb!$ git reset --hard HEAD^!。其中HEAD代表当前版本,\^就代表上一个版本——以此类推\verb!$ git reset --hard HEAD^^!自然就是回退到两个版本啦。

现在我们相当于从21世纪回到了19世纪,那我们要如何再从19实际回到21世纪呢?这时候刚才看起来烦人的commit id就有大用了。首先我们使用\verb!$ git reflog !命令来查看各个提交版本的commit id:
\begin{lstlisting}[language=bash]
wolf-tungsten@wolftungsten-ThinkPad-T460p:~/NotePapers$ git reflog
a0a5fe4 HEAD@{0}: commit: 1
047711e HEAD@{1}: commit: add a line
41320d5 HEAD@{2}: commit (initial): first time
\end{lstlisting}
然后\verb!$ git reset --hard 047711e!,就会回到这个版本。这样的话,时光机就可以任意穿梭了。

\subsection{撤销修改}
首先我们必须了解git的\textbf{工作区}和\textbf{暂存区}:我们实际操作的文件处于工作区,add命令的执行就是将工作区的内容添加到了暂存区,commit命令将暂存区的内容正式提交至版本仓库。

git的存在相当于给文件修改添加了撤销功能,\verb!$ git checkout -- 文件名或者目录名 ! 可以让文件恢复到最近一次commit或者add的状态。
如果改动已经提交到了暂存区,就要使用\verb!$ git reset HEAD 文件名或者目录名! 把文件撤回到工作区进一步修改之后再进行add和commit的操作。

\begin{list}{场景}{\textbf{撤销更改的几种情况}}
\item[场景1]当你改乱了工作区某个文件的内容,想直接丢弃工作区的修改时,用命令git checkout -- file。
\item[场景2]当你不但改乱了工作区某个文件的内容,还添加到了暂存区时,想丢弃修改,分两步,第一步用命令git reset HEAD file,就回到了场景1,第二步按场景1操作。
\item[场景3]已经提交了不合适的修改到版本库时,想要撤销本次提交,参考版本回退一节,不过前提是没有推送到远程库。
\end{list}

\subsection{删除文件}
删除文件是一种常见操作。简单的来说我们可以直接在工作区删除一个文件,然后作为一个新版本提交到版本库,这样我们可以随时利用前面的版本回退,或者撤销更改的方法去“一键还原”。

或者,我们可以采用\verb! $ git rm 文件或者目录 !命令直接从工作区、版本库删除这个文件。

如果发现删错了,那么还可以用\verb!$ git checkout --文件或目录 !将其恢复为版本库中最新的版本。




\end{document}

