\documentclass[UTF8]{ctexart}
\usepackage{fontspec}
\defaultfontfeatures{Mapping=tex-text}
\usepackage{xunicode}
\usepackage{xltxtra}
\usepackage{amsmath}
\usepackage{amsfonts}
\usepackage{amssymb}
\usepackage{graphicx}
\usepackage[left=2cm,right=2cm,top=2cm,bottom=2cm]{geometry}
\author{狼剩子}
\title{\zihao{0}\heiti Git学习笔记}
\newenvironment{myquote}
  {\begin{quote}\kaishu\zihao{-5}}
  {\end{quote}}
  


\usepackage{listings}
  \usepackage{courier}
\lstset{
         basicstyle=\footnotesize\ttfamily, % Standardschrift
         %numbers=left,               % Ort der Zeilennummern
         numberstyle=\tiny,          % Stil der Zeilennummern
         %stepnumber=2,               % Abstand zwischen den Zeilennummern
         numbersep=5pt,              % Abstand der Nummern zum Text
         tabsize=2,                  % Groesse von Tabs
         extendedchars=true,         %
         breaklines=true,            % Zeilen werden Umgebrochen
         keywordstyle=\color{red},        
         stringstyle=\color{white}\sunfamily, % Farbe der String
         showspaces=false,           % Leerzeichen anzeigen ?
         showtabs=false,             % Tabs anzeigen ?
         xleftmargin=17pt,
         framexleftmargin=17pt,
         framexrightmargin=5pt,
         framexbottommargin=4pt,
         %backgroundcolor=\color{thid},
         showstringspaces=false,      % Leerzeichen in Strings anzeigen ?  
         flexiblecolumns     
}
\renewcommand{\lstlistingname}{CODE}
\lstloadlanguages{% Check Dokumentation for further languages ...
         %[Visual]Basic
         %Pascal
         %C
         %C++
         %XML
         %HTML
         Bash
}
  \usepackage{caption}
\DeclareCaptionFont{white}{\color{white}}
\DeclareCaptionFormat{listing}{\colorbox{main}{\parbox{\textwidth}{#1#2#3}}}
\captionsetup[lstlisting]{format=listing,justification=raggedright,labelfont=white,textfont=white, singlelinecheck=false, margin=0pt, font={sf,bf,footnotesize}}




\begin{document}
\maketitle
{\center 教程蓝本:《廖雪峰的git教程》

网址:http://www.liaoxuefeng.com/wiki/0013739516305929606dd18361248578c67b8067c8c017b000}
\section{安装git}
一提到git,大佬们常说:
\begin{myquote}
\kaishu “Git是目前世界上最先进的分布式版本控制系统”
\end{myquote}





所以我们必须理解Git实际上是一个软件,同样需要安装。

在Ubuntu上安装Git的命令:\lstinline!$sudo apt-get install git! 。(Windows和Mac上面的安装很简单,问度娘就好了)
安装完成以后还需要配置账户信息
\begin{lstlisting}[language=bash]
$ git config --global user.name "YourName"
$ git config --global user.email "email@example.com"
\end{lstlisting}

\section{创建版本库}
在需要Git管理的目录下执行命令:
\begin{lstlisting}[caption=bash command]
$ git init
\end{lstlisting}
对于需要Git管理的目录需要使用add命令进行处理
\begin{lstlisting}[caption=bash command]
$ git add 文件名或者目录名
\end{lstlisting}
最后使用commit命令即可提交到版本库
\begin{lstlisting}[caption=bash command]
$ git commit -m “用于识别的附加在提交后面的消息”
\end{lstlisting}

\section{Git时光机}
\textbf{git status}命令可以让我们时刻掌握仓库当前的状态,上面的命令告诉我们,readme.txt被修改过了,但还没有准备提交的修改。

\textbf{git diff}顾名思义就是查看difference,显示的格式正是Unix通用的diff格式,可以从上面的命令输出看到,我们在第一行添加了一个“distributed”单词。

当我们对文件修改后再提交时,也要经过添加新的文件的add和commit命令。

\subsection{版本回退}
在Git中,可以用git log命令查看提交的版本信息,例如在编写个笔记过程中使用的log是这样的:
\begin{lstlisting}
commit a0a5fe4e08f77bc5bf609b27575206cb7cb1ccb2
Author: Acinonyx Tungsten <gaoruihao@outlook.com>
Date:   Thu Feb 9 19:54:30 2017 +0800

    1

commit 047711edf118019cd76d9628d3d61b6022f6b008
Author: Acinonyx Tungsten <gaoruihao@outlook.com>
Date:   Thu Feb 9 08:39:34 2017 +0800

    add a line

commit 41320d5047bf318ab067352bfaa5ce658646d188
Author: Acinonyx Tungsten <gaoruihao@outlook.com>
Date:   Thu Feb 9 08:33:59 2017 +0800

    first time
\end{lstlisting}

这样看起来比较烦的话可以用\verb! $ git log --pretty=oneline!,得到的结果是这样的:
\begin{lstlisting}
wolf-tungsten@wolftungsten-ThinkPad-T460p:~/NotePapers$ git log --pretty=oneline
a0a5fe4e08f77bc5bf609b27575206cb7cb1ccb2 1
047711edf118019cd76d9628d3d61b6022f6b008 add a line
41320d5047bf318ab067352bfaa5ce658646d188 first time
\end{lstlisting}

前面一串串的数字就是十六进制表示的版本号了(官方名称是commit id)。当我们想回退到上一个版本的时候,就可以用命令\verb!$ git reset --hard HEAD^!。其中HEAD代表当前版本,\^就代表上一个版本——以此类推\verb!$ git reset --hard HEAD^^!自然就是回退到两个版本啦。

现在我们相当于从21世纪回到了19世纪,那我们要如何再从19实际回到21世纪呢?这时候刚才看起来烦人的commit id就有大用了。首先我们使用\verb!$ git reflog !命令来查看各个提交版本的commit id:
\begin{lstlisting}[caption=bash command]
wolf-tungsten@wolftungsten-ThinkPad-T460p:~/NotePapers$ git reflog
a0a5fe4 HEAD@{0}: commit: 1
047711e HEAD@{1}: commit: add a line
41320d5 HEAD@{2}: commit (initial): first time
\end{lstlisting}
然后\verb!$ git reset --hard 047711e!,就会回到这个版本。这样的话,时光机就可以任意穿梭了。

\subsection{撤销修改}
首先我们必须了解git的\textbf{工作区}和\textbf{暂存区}:我们实际操作的文件处于工作区,add命令的执行就是将工作区的内容添加到了暂存区,commit命令将暂存区的内容正式提交至版本仓库。

git的存在相当于给文件修改添加了撤销功能,\verb!$ git checkout -- 文件名或者目录名 ! 可以让文件恢复到最近一次commit或者add的状态。
如果改动已经提交到了暂存区,就要使用\verb!$ git reset HEAD 文件名或者目录名! 把文件撤回到工作区进一步修改之后再进行add和commit的操作。

\begin{list}{场景}{\textbf{撤销更改的几种情况}}
\item[场景1]当你改乱了工作区某个文件的内容,想直接丢弃工作区的修改时,用命令git checkout -- file。
\item[场景2]当你不但改乱了工作区某个文件的内容,还添加到了暂存区时,想丢弃修改,分两步,第一步用命令git reset HEAD file,就回到了场景1,第二步按场景1操作。
\item[场景3]已经提交了不合适的修改到版本库时,想要撤销本次提交,参考版本回退一节,不过前提是没有推送到远程库。
\end{list}

\subsection{删除文件}
删除文件是一种常见操作。简单的来说我们可以直接在工作区删除一个文件,然后作为一个新版本提交到版本库,这样我们可以随时利用前面的版本回退,或者撤销更改的方法去“一键还原”。

或者,我们可以采用\verb! $ git rm 文件或者目录 !命令直接从工作区、版本库删除这个文件。

如果发现删错了,那么还可以用\verb!$ git checkout --文件或目录 !将其恢复为版本库中最新的版本。

\section{分支管理}
\subsection{创建与合并分支}
这里以创建dev分支并合并到master分支并最后将其删除为例进行讲解:
\begin{itemize}
\zihao{5}
\item 创建dev分支并切换到dev分支:\verb!$ git checkout -b dev!。checkout命令加上-b参数表示创建和切换。现在可以随心所欲的对dev分支进行修改,而不影响master分支的情况。
\item 通过\verb!$ git branch!命令可以查看现有的分支。
\item 知道了有哪些分支以后就可以再次使用\verb!$ git checkout 分支名!来切换到需要的分支。
\item 合并分支\footnote{需要注意一般情况下是把超前的分支(本例中的dev分支)合并到滞后的分支上(本例中的master分支)。} 首先我们切换到master分支\verb!$ git checkout master!,然后使用\verb!$ git merge dev!命令最终把dev合并到当前分支上。
\item 合并完成后就可以把dev分支删除\verb!$ git branch -d dev!。
\end{itemize}
由于创建dev分支后,没有再对master分支进行修改,所以合并分支的时候git自动为我们使用了“fast-forward”。后面会说明不能Fast-forward的情况以及解决办法。

\subsection{冲突的产生与解决}
在上一节中,创建了dev分支,之后所有的更改都是在dev分支上进行的。下面就要来处理:“创建dev分支后,又对master分支修改”的情况。如果直接进行合并,就会产生冲突。
\begin{lstlisting}
wolf-tungsten@wolftungsten-ThinkPad-T460p:~/test$ git merge dev
自动合并 test
冲突(内容):合并冲突于 test
\end{lstlisting}
这时就要使用\verb!$ git status!命令来查看究竟是哪个文件出了问题

\begin{lstlisting}
wolf-tungsten@wolftungsten-ThinkPad-T460p:~/test$ git status
位于分支 master
您有尚未合并的路径。
  (解决冲突并运行 "git commit")

未合并的路径:
  (使用 "git add <文件>..." 标记解决方案)

	双方修改:   test

修改尚未加入提交(使用 "git add" 和/或 "git commit -a")
\end{lstlisting}
打开冲突的test文件你会发现它变成了这样:
\begin{lstlisting}[caption=bash command]
first
<<<<<<< HEAD
add on master
=======
add in dev
>>>>>>> dev
\end{lstlisting}
这其中用\verb!<<<<<<<\=======\>>>>>>>!标记出来的就是冲突的地方,按照需求修改再保存就可以正确合并了。

再次提交后,用带参数的git log可以查看合并的情况:
\begin{lstlisting}[caption=bash command]
wolf-tungsten@wolftungsten-ThinkPad-T460p:~/test$ git log --graph --pretty=oneline 
--abbrev-commit
*   0ee2246 2on master
|\  
| * 0cb0c0b 2 on dev
* | 99d0183 commit on master
|/  
* c59e421 1
\end{lstlisting}

\subsection{禁用Fast Forward}
合并的时候如果可以使用fastforward固然很爽,但是带来的代价是,一旦我们删除分支,那么分支记录就完全丢失了。这里可以在合并的时候加入\verb!--no-ff!参数来禁用fastforward。

完整的命令:\verb!git merge --no-ff -m "merge with no-ff" dev!。因为合并后会产生一个新的提交,所以还是需要-m补充提交的信息。实际开发过程应该尊崇如下的管理规则:
\begin{myquote}
在实际开发中,我们应该按照几个基本原则进行分支管理:
\begin{enumerate}
\item master分支应该是非常稳定的,也就是仅用来发布新版本,平时不能在上面干活;
\item 干活都在dev分支上,也就是说,dev分支是不稳定的,到某个时候,比如1.0版本发布时,再把dev分支合并到master上,在master分支发布1.0版本;
\item 每个人都在dev分支上干活,每个人都有自己的分支,时不时地往dev分支上合并就可以了
\end{enumerate}
\end{myquote}

\subsection{使用stash保护工作现场}
适用于:工作进行到一半还不适合提交,但是有需要临时去处理别的版本或者分支上的问题。
首先使用\verb!$ git stash!命令,这时候工作区已经保存成了快照,可以放心的切换版本切换分支干其他的事情。然后,等到其他问题处理完成后,用\verb!$ git stash list!罗列以保存的工作区。
\begin{lstlisting}[caption=bash command]
$ git stash list
stash@{0}: WIP on dev: 6224937 add merge
\end{lstlisting}

接下来有两种选择:
\begin{itemize}
\item 切换回工作环境:使用\verb!git stash apply!,这样切换回去以后并不会删除保存的快照,还需要\verb!git stash drop!来删除保存的快照。
\item 使用\verb!git stash pop!,恢复的同时也把保存的快照删掉了。
\end{itemize}

还有更美好的事情,想我这种不爱收拾桌子的人,可以保存多个快照。需要恢复现场的时候用\verb!$ git stash list!罗列,再用\verb!$ git stash apply stash@{0}!恢复到指定状态就可以啦。

\subsection{强行删除一个分支}
前面说过,用\verb!git branch -d 分支!可以删除一个分支。但是这样操作的前提条件是这个分支已经合并到主分支上去了。但是有的时候可能需要“甩锅”————还没开发完成就放弃。这样的分支是不能合并到dev上去,更不能合并到master上去。要删除一个未合并的分支要用\verb!git branch -D 分支!(大写D)。

\section{标签的使用}
标签和commit是相互对应的,它的用途简单来说就commit id不便于记忆和识别,给它起个别名。
\subsection{打标签}
使用命令\verb!$ git tag <标签名称>!就直接给当前分支最新的版本打上了标签。

使用命令\verb!$ git tag <标签名称> <commit_id>!可以给指定的commit打标签。

使用命令\verb!$ git tag -a <标签名称> -m <文字说明> <commit id>!来打一个带文字说明的标签。

需要查看有哪些标签的时候可以\verb!git tag!,就会列出所有打过的标签\footnote{使用这个命令列出的标签是按照字母顺序排序的,不是按照时间顺序的。} 

\subsection{撕标签}
标签打错了就比较尴尬了,不过还是有办法缓解尴尬。

\verb!$ git tag -d <标签名称>!删除掉指定的标签。

标签默认是储存在本地的,不会推送到远程版本库。如果很任性的话也可以\verb!$ git push origin <标签名称>!推送到远程库。或者一次性推送所有标签\verb!$ git push origin --tags!。一旦推送到远程,想要删除就比较麻烦:首先删除本地标签;再用\verb!$ git push origin :refs/tags/<标签名称>!删除远程标签。


\section{远程仓库以及github的使用}
\subsection{如何和github产生py关系}
\begin{itemize}
\item 执行bash命令\verb!ssh-keygen -t rsa -C "youremail@example.com"!,然后就可以在home目录下找到一个.ssh 文件夹,其中包含有一个\verb!id_rsa.pub!文件,复制其中的全部内容。
\item 登陆GitHub,打开“Account settings”,“SSH Keys”页面,点“Add SSH Key”,填上任意Title,在Key文本框里粘贴\verb!id_rsa.pub!文件的内容,点“Add Key”就可以了。
\end{itemize}
\subsection{克隆远程仓库到本地}
在本地要存放仓库的地方使用命令\verb!$ git clone git@github.com:用户名/仓库名.git!。可以找到当前操作目录下面多出来一个以仓库名命名的文件夹,实际上就是完整的git仓库。



\end{document}