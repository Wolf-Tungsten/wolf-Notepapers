\documentclass[cyan]{elegantnote}

\author{狼剩子}
\email{gaoruihao@outlook.com}
\zhtitle{iGuideCar项目}
\entitle{Project iGuideCar}
\version{1}
\myquote{\center 靡不有初{ }鲜克有终}
\logo{empty.jpg}
\cover{empty.jpg}

%green color
   \definecolor{main1}{RGB}{210,168,75}
   \definecolor{seco1}{RGB}{9,80,3}
   \definecolor{thid1}{RGB}{0,175,152}
%cyan color
   \definecolor{main2}{RGB}{239,126,30}
   \definecolor{seco2}{RGB}{0,175,152}
   \definecolor{thid2}{RGB}{236,74,53}
%cyan color
   \definecolor{main3}{RGB}{127,191,51}
   \definecolor{seco3}{RGB}{0,145,215}
   \definecolor{thid3}{RGB}{180,27,131}

\usepackage{makecell}
\usepackage{lipsum}

%颜色的切换使用 \color{main} \color{seco} \color{thid}


\usepackage{listings}
  \usepackage{courier}
\lstset{
         basicstyle=\footnotesize\ttfamily, % Standardschrift
         %numbers=left,               % Ort der Zeilennummern
         numberstyle=\tiny,          % Stil der Zeilennummern
         %stepnumber=2,               % Abstand zwischen den Zeilennummern
         numbersep=5pt,              % Abstand der Nummern zum Text
         tabsize=2,                  % Groesse von Tabs
         extendedchars=true,         %
         breaklines=true,            % Zeilen werden Umgebrochen
         keywordstyle=\color{red},        
         stringstyle=\color{white}\sunfamily, % Farbe der String
         showspaces=false,           % Leerzeichen anzeigen ?
         showtabs=false,             % Tabs anzeigen ?
         xleftmargin=17pt,
         framexleftmargin=17pt,
         framexrightmargin=5pt,
         framexbottommargin=4pt,
         %backgroundcolor=\color{thid},
         showstringspaces=false,      % Leerzeichen in Strings anzeigen ?  
         flexiblecolumns   
}
\renewcommand{\lstlistingname}{CODE}
\lstloadlanguages{% Check Dokumentation for further languages ...
         %[Visual]Basic
         %Pascal
         %C
         %C++
         %XML
         %HTML
         Bash
}
  \usepackage{caption}
\DeclareCaptionFont{white}{\color{white}}
\DeclareCaptionFormat{listing}{\colorbox{main}{\parbox{\textwidth}{#1#2#3}}}
\captionsetup[lstlisting]{format=listing,justification=raggedright,labelfont=white,textfont=white, singlelinecheck=false, margin=0pt, font={sf,bf,footnotesize}}




\begin{document}
\maketitle
\tableofcontents
\chapter{服务器的配置}
目标使用Nginx+uwsgi+Django方式搭建,为微信公众平台和硬件提供接口。
\section{相关服务的配置}
\subsection{编译安装 python3.6.0}
\subsection{Nginx的安装及使用}

我在阿里云使用的是CentOS7,所以下面的步骤都是按照这个系统的操作来进行的,对于Ubuntu/Debian系列的系统可能会在后续作补充。

\textbf{Nginx的安装}:\verb!$ yum install -y nginx!。不出意外一步搞定。

\textbf{Nginx配置}:
配置文件位于\verb!\etc\nginx\nginx.conf!;下面抽取文件最关键部分(基本上按照下面这个修改就可以了)。
\begin{lstlisting}[language=bash,caption=nginx.conf]
server {
        listen       80 default_server;
        listen       [::]:80 default_server;
        server_name  _;
        index index.html;
        root        /root/webroot/;

        # Load configuration files for the default server block.
        include /etc/nginx/default.d/*.conf;

        location / {
                root /root/webroot/page;
                index index.html;
        }

\end{lstlisting}


\textbf{需要修改的内容如下}
\begin{itemize}
\item 第一行添加\verb!user root;!
\item server下面两个listen后面的80指端口号,按照需要修改
\item 第一个root指定网站的根目录
\item 下面location就是指定URL后面的path了
\item 也需要指定root属性
\item index属性指定访问当前URL默认的页面
\end{itemize}

\textbf{服务的启动、停止、重新载入}
\begin{itemize}
\item \textbf{启动服务}\verb!# service nginx start!
\item \textbf{重启或者重新载入配置文件}\verb!# service nginx restart !,或者\verb!# nginx -s reload !
\item \textbf{停止服务}\verb!# service nginx stop!,或者\verb!# nginx -s quit!
\end{itemize}


\subsection{uWSGI的安装及使用}
\textbf{简易步骤}

uWSGI的安装(简单粗暴):\verb|sudo pip install uwsgi|

使uWSGI和Django项目产生py关系:
\begin{enumerate}
\item 在Django项目目录(就是存放有manage.py的目录)下面新建一个配置文件\verb|uwsgi.ini|,文件内容如下:
\begin{lstlisting}[language=bash,caption=uwsgi.ini]
[uwsgi] 
chdir=/path/to/your/project 包含manage.py 的文件夹 
module=mysite.wsgi:application  
master=True 
pidfile=/tmp/project-master.pid 
vacuum=True 
max-requests=5000 
socket=127.0.0.1:5000
\end{lstlisting}
\item 然后在当前目录下使用命令:\verb|$ uwsgi --ini uwsgi.ini --http(此处应有一个空格) :端口|
\item 如果遇到端口被占用的情况:首先\verb!$ netstat -lnp|grep 端口号 !获取到占该端口进程的pid,然后\verb|$ kill -9 pid |干掉就可以了。
\end{enumerate}
\subsection{Nginx与uwsgi链接}
首先创建\verb|uwsgi_params|文件,文件内容如下:
\begin{lstlisting}[language=bash,caption=uwsgi]
uwsgi_param  QUERY_STRING       $query_string;
uwsgi_param  REQUEST_METHOD     $request_method;
uwsgi_param  CONTENT_TYPE       $content_type;
uwsgi_param  CONTENT_LENGTH     $content_length;

uwsgi_param  REQUEST_URI        $request_uri;
uwsgi_param  PATH_INFO          $document_uri;
uwsgi_param  DOCUMENT_ROOT      $document_root;
uwsgi_param  SERVER_PROTOCOL    $server_protocol;
uwsgi_param  REQUEST_SCHEME     $scheme;
uwsgi_param  HTTPS              $https if_not_empty;

uwsgi_param  REMOTE_ADDR        $remote_addr;
uwsgi_param  REMOTE_PORT        $remote_port;
uwsgi_param  SERVER_PORT        $server_port;
uwsgi_param  SERVER_NAME        $server_name;
\end{lstlisting}

配置nginx,修改\verb!\etc\nginx\nginx.conf!
\begin{lstlisting}[language=bash,caption=和uwsgi建立了关系的nginx配置]
server {
        listen       80;
        server_name  112.74.49.113;
        index index.html;
        root        /root/webroot/;

        # Load configuration files for the default server block.
        include /etc/nginx/default.d/*.conf;

        location / {
                include /root/iGuideCar/uwsgi_params;
                uwsgi_pass 127.0.0.1:5000;
        }
\end{lstlisting}
\begin{itemize}
\item include后面指向刚才写好的uwsgi\_params
\item uwsgi pass 后面填写的内容\textbf{一定必须千千万万}和uwsgi.ini中写的socket\textbf{一毛一样}(就被这个地方坑了一晚上)
\item 每个行后面要写分号(这个也坑了一下)
\end{itemize}

启动uwsgi服务,启动nginx服务就搞定了QAQ。
\end{document}
